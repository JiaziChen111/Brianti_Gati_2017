\documentclass{beamer}
\usepackage[english]{babel}
\usepackage[latin1]{inputenc}
\usepackage[T1]{fontenc}
\usepackage{amssymb}
\usepackage{amsmath}
\usepackage{booktabs}
\usepackage{verbatim}
\usepackage{caption}
\usepackage{float}
\usepackage{csquotes}
\usepackage{sansmathaccent}
\usepackage{subfigure}
\usepackage{multicol}
\pdfmapfile{+sansmathaccent.map}
\def \ourFigPath {../../} 
\def \ourTablePath {../../Tables/} 

\setbeamersize{text margin left=5mm,text margin right=12mm} 
%
\font\reali=msbm10 at 12pt
% subsets of real numbers
\newcommand{\numberset}{\mathbb}
\newcommand{\real}{\hbox{\reali R}}
\newcommand{\N}{\numberset{N}}
\newcommand{\realp}{\hbox{\reali R}_{\scriptscriptstyle +}}
\newcommand{\realpp}{\hbox{\reali R}_{\scriptscriptstyle ++}}
\newcommand{\virgolette}[1]{``#1''}
%

\author[Brianti, Gati]{Marco Brianti and Laura Gati}

\institute[Boston College]{Boston College}


\title{Title}

\date{\today}

\usetheme{Warsaw}


\begin{document}


\begin{frame}

\maketitle


\end{frame}


%%%%%%% Slide %%%%%%
\begin{frame}
\frametitle{A two-sector model}

Final goods firm (GE)
\begin{align*}
\max_{K_{1,t}(s), L_{1,t}(s), S^{IT}_t(s) } Y_t(s) =  & K_{1,t}(s)^{\alpha}L_{1,t}(s)^{\beta}S^{IT}_t(s)^{1-\alpha - \beta}A_t \\
 & - w_t L_{1,t}(s) -R_tK_{1,t}(s) -P^{IT}_t IT_t(s) \\
 & \text{where} \; S^{IT}_t(s) =  (1-\delta^{IT})S^{IT}_{t-1} + IT_t \\
 &  \; A_t =  \eta_t \Psi_t (G(S^{IT}_t,S^{IT}_{t-1},\dots))^{\gamma} \\
 &  \; G(S^{IT}_t,S^{IT}_{t-1},\dots) = \zeta_1 S_t^{IT} + \zeta_2 S^{IT}_{t-1} + \dots \\
  & \text{and} \; \sum_{i=1}^{N} \zeta_i = 1
\end{align*}

\end{frame}

\begin{frame}

IT-producing firm (Google)
\begin{align*}
\max_{K_{2,t}(s), L_{2,t}(s)} IT_t(s) = P^{IT}_t \eta_t \lambda_t F(K_{2,t}(s), L_{2,t}(s)) -w_t L_{2,t}(s) - R_t K_{2,t}(s)
\end{align*}

\

A news shock is $\eta_{t+k} \uparrow$ for some positive $k$. 

An IT productivity shock is $\lambda_{t} \uparrow$, i.e. an increase in IT productivity today. 

$\Psi$ is a final-good-specific exogenous technology process.

\end{frame}
%%%%%%%%%%%%%%%%%

%%%%%%% Slide %%%%%%
\begin{frame}
\frametitle{Proposition 1 - the Solow-residual in this model}

\begin{align*}
\frac{\dot{A}}{A} = & \frac{\dot{Y}}{Y} - \alpha \frac{\dot{K_1}}{K_1} - \beta \frac{\dot{L_1}}{L_1} - (1-\alpha - \beta) \frac{\dot{S^{IT}}}{S^{IT}} \\
= & \frac{\dot{\eta}}{\eta} + \frac{\dot{\Psi}}{\Psi} + \gamma \frac{\dot{G}(S^{IT},\dots)}{G(S^{IT},\dots)}
\end{align*}

TFP consists of an exogenous, common technology process $\eta$, a final-good-specific exogenous technology process $\Psi$ and an endogenous component coming as a spillover from the aggregate stock of IT technology. 
\end{frame}
%%%%%%%%%%%%%%%%%

%%%%%%% Slide %%%%%%
\begin{frame}
\frametitle{Proposition 2 - when a news shock leaves relative prices constant}

Since the rate of return on the input factors have to equal in the two sectors, we can set the FOCs from GE and Google equal. Let's do this for capital:

\begin{align*}
R_t = & \alpha \frac{Y_t(s)}{K_{1,t}(s)} P^C_t = \eta_t \lambda_t F_k P^{IT}_t \\
\Leftrightarrow \frac{P^{IT}_t }{P^{C}_t } = & \alpha \frac{K_{1,t}(s)^{\alpha}L_{1,t}(s)^{\beta}S^{IT}_t(s)^{1-\alpha - \beta}\eta_t \Psi_t (S^{IT}_t)^{\gamma} }{K_{1,t}(s)} \frac{1}{\eta_t \lambda_t F_k}
\end{align*}


A news shock doesn't move relative prices if 
\begin{itemize}
\item the news shock doesn't change the relative marginal productivities of the two inputs in the two sectors, so that labor and capital are not reallocated from one sector to another.
\end{itemize}

\end{frame}
%%%%%%%%%%%%%%%%%

%%%%%%% Slide %%%%%%
\begin{frame}
\frametitle{Evidence that $\gamma \neq 0$; GPTs show up as spillovers}

Theoretical evidence
\begin{itemize}
\item Whelan (2001): a two-sector model with IT relative prices.
\item Brynjolfsson, e.g. in Bloom et al. (2014): theory of IT as a GPT changing management practices.
\item Mokyr (1990):  an economic history account of how GPTs (electricity, computers) change the \textbf{organization} of production $\rightarrow$ neither use, nor output value of IT, but \textbf{indirect} effect
\item Romer (1986): a production function with spillovers from human capital.
\end{itemize}

Empirical evidence
\begin{itemize}
\item Basu, Fernald, Oulton \& Srinivasan (2004 Macro Annuals)
\item Stiroh (2002 AER): industry-level data show that productivity gains from IT are not confined to industries producing IT.
\item Oliner \& Sichel (2000 JEP): growth accounting.
\item Black \& Lynch (2004 EJ), Doms, Dunn \& Troske (1997 QJE): plant-level data.
\end{itemize}

\end{frame}
%%%%%%%%%%%%%%%%%






\end{document}
