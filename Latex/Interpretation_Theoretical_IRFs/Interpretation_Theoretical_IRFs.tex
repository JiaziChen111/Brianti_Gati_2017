\documentclass{article}
\usepackage[utf8]{inputenc}
\usepackage[english]{babel}
\usepackage[a4paper,top=3.5cm,bottom=3.5cm,left=3.5cm,right=3.5cm,%
bindingoffset=0mm]{geometry}
\usepackage{amssymb}
\usepackage{amsmath}
\newtheorem{prop}{Proposition}
\newtheorem{lemma}{Lemma}
\newenvironment{proof}[1][Proof]{\begin{trivlist}
		\item[\hskip \labelsep {\bfseries #1}]}{\end{trivlist}}
\newcommand{\qed}{\nobreak \ifvmode \relax \else
	\ifdim\lastskip<1.5em \hskip-\lastskip
	\hskip1.5em plus0em minus0.5em \fi \nobreak
	\vrule height0.75em width0.75em depth0em\fi}
\usepackage{tikz}
\usepackage{graphicx}
\usepackage{rotating}
\usepackage{float}
\linespread{1.3}
\raggedbottom




%
\font\reali=msbm10 at 12pt
% subsets of real numbers
\newcommand{\real}{\hbox{\reali R}}
\newcommand{\realp}{\hbox{\reali R}_{\scriptscriptstyle +}}
\newcommand{\realpp}{\hbox{\reali R}_{\scriptscriptstyle ++}}
\newcommand{\R}{\mathbb{R}}
\DeclareMathOperator{\E}{\mathbb{E}}
%

\title{Theoretical Impulse Response Functions}
\author{Marco Brianti\\Laura Gati}
\date{October 2017}


\begin{document}
	
	\maketitle
	
	\section{Zero Balanced-Growth Path}
	
	Define $x_t$ as the value of variable $x$ at time $t$. Define $\overline{x}$ as the value of variable $x$ in steady state. Impulse response functions are percentage deviation of $x$ from its steady state value $\overline{x}$. Formally,
$$
IRF(x_t | \varepsilon) = \log \bigg( \frac{x_t}{\overline{x}}  \bigg)
$$
where $\varepsilon$ is a general perturbation from the steady state level of the system. As long as $\varepsilon$ has a transitory nature, we expect IRFs to converge to zero asymptotically. In other words, we expect that the system will naturally converge to the steady state level once the perturbation ceases.
	
\section{Positive Balanced-Growth Path - Level Shock}

Define $x_t$ as the value of variable $x$ at time $t$. Define $\tilde{x}_t$ as the value of $x$ when divide by the value of an exogenous process which is growing at the same rate of $x$. Formally,
$$
\tilde{x}_t = \frac{x_t}{G_{t-1}}
$$
where $G_{t-1}$ may represent the value of the exogenous productivity process which is growing at the same rate of $x$. Define $\tilde{x}$ as the value of $\tilde{x_t}$ when the system is growing at the balanced growth at the BGP without any past or present source of perturbation. In this case, impulse response functions are percentage deviation of $\tilde{x}_t$ from its steady state value $\tilde{x}$. Formally,
\begin{eqnarray}
\begin{aligned}
IRF(\tilde{x}_t | \varepsilon) &= \log \bigg( \frac{\tilde{x}_t}{\tilde{x}}  \bigg)  \\
                               &= \log \bigg( \frac{\frac{x_t(\varepsilon)}{G_{t-1}}}{\frac{x_t}{G_{t-1}}}  \bigg)  \\
                               &= \log \bigg( \frac{x_t(\varepsilon)}{x_t} \bigg)  \\
\end{aligned}
\end{eqnarray}
In other words, we can interpret $IRF(\tilde{x}_t | \varepsilon)$ as percentage deviation of $x_t$ from the value it should have had without $\varepsilon$. Also in this case we expect IRFs to converge to zero asymptotically. In other words, we expect $x_t(\varepsilon)$ to converge back to $x_t$ once the perturbation ceases.

\section{Positive Balanced-Growth Path - Trend Shock}

Define $x_t$ as the value of variable $x$ at time $t$. Define $\tilde{x}_t$ as the value of $x$ when divide by the value of an exogenous process which is growing at the same rate of $x$. Formally,
$$
\tilde{x}_t = \frac{x_t}{G_{t-1}}
$$
where $G_{t-1}$ may represent the value of the exogenous productivity process which is growing at the same rate of $x$. Define $\tilde{x}$ as the value of $\tilde{x}_t$ when the system is growing at the balanced growth at the BGP without any past or present source of perturbation. In this case, impulse response functions are percentage deviation of $\tilde{x}_t$ from its steady state value $\tilde{x}$. Formally,
\begin{eqnarray}
\begin{aligned}
IRF(\tilde{x}_{t+h} | \varepsilon) &= \log \bigg[ \Pi_{i=0}^h \bigg( \frac{\tilde{x}_{t+i}(\varepsilon)}{\tilde{x}_{t+i-1}(\varepsilon)}  \frac{G_{t+i-1}}{G_{t+i-2}}  \bigg) \bigg]  \\
&= \log \Bigg[ \Pi_{i=0}^h \Bigg( \frac{\frac{x_{t+i}(\varepsilon)}{G_{t+i-1}(\varepsilon)}}{\frac{x_{t+i-1}(\varepsilon)}{G_{t+i-2}(\varepsilon)}}  \frac{G_{t+i-1}(\varepsilon)}{G_{t+i-2}(\varepsilon)}  \Bigg) \Bigg]  \\
&= \log \Bigg[ \Pi_{i=0}^h \Bigg( \frac{x_{t+i}(\varepsilon)}{x_{t+i-1}(\varepsilon)}   \Bigg) \Bigg]  \\
&= \log \bigg( \frac{x_{t+h}(\varepsilon)}{x_{t-1}}  \bigg)  \\
\end{aligned}
\end{eqnarray}

	

\end{document}
