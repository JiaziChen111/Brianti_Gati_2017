\documentclass[14pt]{article}

% Packages
\usepackage[utf8]{inputenc}
\usepackage[english]{babel}
\usepackage[top=3cm,bottom=3cm,left=3cm,right=3cm,bindingoffset=0mm]{geometry}
\usepackage{amssymb}
\usepackage{amsmath}
\usepackage{tikz}
\usepackage{graphicx}
\usepackage{comment}
\usepackage{rotating}
\usepackage{float}
\usepackage{natbib}
\usepackage{amsthm}
\usepackage{bbm}
\usepackage{thmtools,thm-restate}
\usepackage{hyperref}
\usepackage{extsizes}
\usepackage[font=footnotesize,labelfont=bf]{caption}

% New Options
\newtheorem{prop}{Proposition}
\newtheorem{definition}{Definition}[section]
\newtheorem*{remark}{Remark}
\newtheorem{lemma}{Lemma}
\declaretheorem{proposition}
\linespread{1.3}
\raggedbottom
\font\reali=msbm10 at 12pt

% New Commands
\newcommand{\real}{\hbox{\reali R}}
\newcommand{\realp}{\hbox{\reali R}_{\scriptscriptstyle +}}
\newcommand{\realpp}{\hbox{\reali R}_{\scriptscriptstyle ++}}
\newcommand{\R}{\mathbb{R}}
\DeclareMathOperator{\E}{\mathbb{E}}

\title{ICT and Future Productivity:\\Evidence and Theory of a GPT\thanks{Correspondence: Department of Economics, Boston College, 140 Commonwealth Avenue, Chestnut Hill, MA 02467. Email: brianti@bc.edu (Marco Brianti) and gati@bc.edu (Laura Gati).}}
\author{Marco Brianti \\ {\small Boston College} \and Laura Gati \\ {\small Boston College}}
\date{\today}


\begin{document}



\maketitle
%%%%%%%%%%%%%%%%%%%%             ABSTRACT           %%%%%%%%%%%%%%%%%% 
\small{
\abstract{Information and Communication technology (ICT) is able to explain accelerations in productivity in sectors that are ICT users. We employ Structural VARs to investigate the effects of ICT supply shocks on Total Factor Productivity (TFP) and other macroeconomic variables. In response to this sector-specific supply shock relative prices of ICT goods and services immediately fall, ICT investment rises on impact, and TFP displays a significant delayed and persistent increase. In line with theories of ICT as a general-purpose technology, we analyze a two-sector general equilibrium model in order to rationalize previous results and estimate key parameters via impulse-response matching. We conclude that ICT accumulation is able to enhance productivity through a positive spillover effect which takes into account the overall level of diffusion of ICT capital in the economy.}
}
\newpage



\newpage
%%%%%%%%%%%%%%%%%%%%               INTRODUCTION                 %%%%%%%%%%%%%%%%%% 
\section{Introduction}

Although there is large consensus on the importance of productivity as a driver of economic performances, less agreement is on the underlying sources that enhance its growth. For several years most of the business-cycle literature purposely decided to avoid such a question by proxying movements in productivity as random shocks.\footnote{\cite{kydland1982time} and \cite{long1983real} are among the first papers which consider productivity shocks on general equilibrium models.} However, the robust empirical evidence of the slowdown in productivity right before the great recession is summoning the literature to take a step back and devote more attention on the drivers of medium-term productivity growth.\footnote{See \cite{cette2016pre} and \cite{byrne2016does} among others.}

Along \cite{comin2006medium}, some theoretical contributions rationalize endogenous productivity dynamics by adapting features of endogenous growth models into DSGE models. Following \cite{romer1990endogenous}, most of those papers augment final-good production functions with an expanding composite of intermediate goods produced by the R\&D sector in order to allow for an endogenous rate of adoption of new technologies.\footnote{\cite{bianchi2014growth}, \cite{anzoategui2016endogenous}, and \cite{moran2017innovation} use similar techniques to endogenize growth. In particular, \cite{bianchi2014growth} augment a DSGE  model using a quality ladders model in the vein of \cite{grossman1991quality}. Moreover, \cite{anzoategui2016endogenous} and \cite{moran2017innovation}, similarly to \cite{comin2006medium}, use a model of expanding variety in the vein of \cite{romer1990endogenous}.} Consistent with those previous models, other papers attempt to provide empirical evidence of a slowdown in the productivity of the R\&D sector. Specifically, they show that although research effort is keeping rising, the rate of new ideas and discoveries is slowing down.\footnote{\cite{jones2009burden} and \cite{bloom2017ideas} are two important contributions that highlight those facts.}

Motivated by this wave of research, this paper follows a different path and argue that Information and Communication Technology (hereafter ICT) plays an important role in driving medium-term productivity in sectors that are ICT users. Our contribution is twofold. First, we provide a robust empirical evidence to show that current rises in ICT investment explains significant and persistent increases in future Total Factor Productivity (hereafter TFP). Second, we analyze a standard theoretical framework in order to both motivate and rationalize our empirical results. 

Regarding the empirical section, the idea is to identify technological shocks which are only specific to the ICT sector in a Structural VAR context.\footnote{An interesting paper which is somehow related to our empirical part is \cite{jafari2012impact}. The authors identify ICT shocks as a potential driver of the Iranian business cycle using a completely different identification strategy and obtaining qualitatively different results.} In order to have a reliable identification procedure our multivariate system needs to embody three key variables: TFP, ICT investment (hereafter ICTI), and relative prices (hereafter RP). Importantly, ICTI is defined as the total expenditure in equipment and computer software meant to be used in production for more than an year. Thus, an increase in ICTI has to be though as an ICT capital deepening. Moreover RP is the ratio between prices in the ICT sector over prices in the overall economy. To verify that we are correctly identifying an ICT technology shock, we firstly expect it to be orthogonal to the productivity of all the other sectors. Since the share of the ICT sector accounts for a negligible part in the whole economy we expect it should have an approximately zero effect on TFP. Moreover, as pointed out by \cite{greenwood1997long} and \cite{fisher2006dynamic}, embodying RP and ICTI is important because we expect that a sectoral technology shock should decrease its relative prices and enhance expenditure in the underlying sector.\footnote{However, as suggested by both \cite{greenwood2000role} and \cite{basu2010sector}, we are aware that conditioning our identifying restrictions only on the direction of RP does not properly measure for technological changes between sectors. This is the main reason why we never impose the direction of RP as an direct identifying condition.} In response to this shock, ICTI rises on impact and remains significant for several quarters. RP persistently and significantly declines for more than two years and TFP, which does not react on impact, rises after few quarters and remains significant and stable for at least 5 years. 

Although our results are robust over different specifications, there is an important critique that our empirical strategy had to carefully take into account, which is the reverse causality due to news on future TFP. A well-taken concern motivated my the news-shock literature is that the positive reaction of ICTI on impact may be triggered to signals related future increases in TFP and not to contemporaneous ICT technological improvements. In other words, our identification strategy may confound a news shock which contemporaneously enhances investment in ICT capital goods. In order to take into account this potential issue, we provide a series of robustness check where sequentially we firstly identify a news shock in the spirit of \cite{barsky2011news} and subsequently we identify our sectoral ICT shock using the previous identification strategy. Encouragingly, controlling for signals regarding future movements in TFP does not affect any conclusion drown so far. In particular, we can now state even more strongly the causality relation from movements in current ICT technological changes to future TFP. 

In order to formally explain which mechanism links current ICT to future TFP, we analyze a 2-sector DSGE model which allows ICT to be the general purpose technology of the whole economy.



\




In line with theories of ICT as a general-purpose technology, we analyze a two-sector general equilibrium model in order to rigorously rationalize previous results and estimate key parameters via impulse-response matching. We conclude that ICT accumulation is able to enhance productivity through a positive spillover effect which takes into account the overall level of diffusion of ICT capital in the economy.





%%%%%%%%%%%%%%%%%%%%%%            EMPIRICS                     %%%%%%%%%%%%%%%%%%%
\section{Empirics}



%%%%%%%%%%%%%%%%%%%%               MODEL                 %%%%%%%%%%%%%%%%%% 
\section{Model}




%%%%%%%%%%%%%%%%%%%%               CONCLUSION                 %%%%%%%%%%%%%%%%%% 
\section{Conclusion}
hjklbhjkl

\bibliographystyle{chicago}
\bibliography{literature}



 %%%%%%%%%%%%%%%%%%           BIBLIOGRAPHY            %%%%%%%%%%%%%%%%%% 

 
\end{document}


